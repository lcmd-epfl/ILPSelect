\documentclass[11pt,a4paper]{article}
\usepackage{anysize}
\usepackage{amsmath}
\usepackage{bbold}
%\usepackage{utf8math}
 
%\DeclareMathOperator{\lp}{\textbf{LP}}
\DeclareMathOperator{\lp}{LP}
\DeclareMathOperator{\supp}{supp}
\newcommand{\A}{\mathcal{A}}
\newcommand{\T}{\mathcal{T}}
\newcommand{\LO}{\mathcal{O}}



\begin{document}


\section{Basic problem}
\label{sec:1}

Given a matrix $A \in \mathbb{R}_{{\geq 0}}^{n \times n}$, representing a \emph{target} molecule and a \emph{database} $B^{(i)} \in \mathbb{R}^{n_i \times n_i}$, $i \in \{1,\dots,k\}$. The $n_i$ are usually small, around $7$. 


A \emph{solution} is a permutation $\pi \in S_n$, a  multiset  $S \subseteq \{1,\dots,k\}$ and a mapping $p: S \to \{1,\dots,n\}$. The \emph{objective value} of the  solution is $\| A_\pi - A_{S,p}\|_2^2$, where $A_\pi$ is the matrix stemming from $A$ where the rows and columns were permuted by $\pi$ and $A_{S_p}$ is the block-diagonal matrix that stems from the elements in $S$ that are put with upper leftmost component on the corresponding position described by $p$.

A feasibility constraint is that $S$ and $p$ induce a partition  of the set $\{1,\dots,n\}$.



Overarching goal for Chemistry: Suggest new ways to synthesize a target molecule.



\begin{itemize}
\item Bigger database of pieces should make sense. Efficiency is a
  problem can be overcome.  Also the pieces need a score (objective
  value) on the cost to synthesize them. This can be modeled in the objective function. 
\item Diversity of solutions: One way would be to list pareto optimal
  solutions using distance, sums of costs of pieces, .... Another way
  would be to model diversity
\item Objective function should also model the number of smaller pieces. The more pieces in the solution, the more synthesis steps.
 
\end{itemize}


\section{Optimal placement of molecules}
The first goal is to find a way to optimally place a small molecule represented by the matrix $M \in \mathbb{R}^{m \times m}$ inside the target represented by the matrix $T \in \mathbb{R}^{n \times n}$. 
In other words, we want to find the mapping $\phi : \{1, \dots, m\} \to \{1, \dots, n\}$. This is an index assignment problem where the indices $x_{ij} \in \{0, 1\}$ are assigned:
\begin{equation*}
    x_{ij} = 
    \begin{cases}
        1 & \text{if } \phi(i) = j \\
        0 & \text{otherwise}
    \end{cases}
\end{equation*}
under the constraints:
\begin{align*}
    \forall i \in \{1, ..., m\} &: \sum_{j=1}^n x_{ij} = 1 \\
    \forall j \in \{1, ..., n\} &: \sum_{i=1}^m x_{ij} \leq 1 
\end{align*}

This consists of minimising the objective function:
\begin{equation*}
    \mathrm{min} \sum (m_{ij} - T_{kl})^2 x_{ik} * x_{jl}
\end{equation*}


\end{document}

